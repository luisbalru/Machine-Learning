\input{preambuloSimple.tex}
\graphicspath{ {./images/} }
\usepackage{subcaption}
\usepackage{hyperref}
\usepackage{soul}


%----------------------------------------------------------------------------------------
%	TÍTULO Y DATOS DEL ALUMNO
%----------------------------------------------------------------------------------------

\title{	
\normalfont \normalsize 
\textsc{\textbf{Aprendizaje Automático (2019)} \\ Doble Grado en Ingeniería Informática y Matemáticas \\ Universidad de Granada} \\ [25pt] % Your university, school and/or department name(s)
\horrule{0.5pt} \\[0.4cm] % Thin top horizontal rule
\huge Memoria Práctica 3 \\ % The assignment title
\horrule{2pt} \\[0.5cm] % Thick bottom horizontal rule
}

\author{Luis Balderas Ruiz \\ \texttt{luisbalderas@correo.ugr.es}} 
 % Nombre y apellidos 


\date{\normalsize\today} % Incluye la fecha actual

%----------------------------------------------------------------------------------------
% DOCUMENTO
%----------------------------------------------------------------------------------------

\begin{document}

\maketitle % Muestra el Título

\newpage %inserta un salto de página

\tableofcontents % para generar el índice de contenidos

\listoffigures

\listoftables

\newpage


%----------------------------------------------------------------------------------------
%	Introducción
%----------------------------------------------------------------------------------------

\section{Recognition of handwritten digits}

\subsection{Introducción}

Nos enfrentamos a un problema de clasificación multietiqueta (9 clases) sobre una base de datos de reconocimiento de dígitos manuscritos, proveniente de la Universidad de Bogazici. Tras extraer mapas de bits de dimensión $32\times32$ normalizados, se dividen en bloques de $4\times4$ disjuntos y se cuenta el número de píxeles en cada bloque. Esto genera una matriz $8\times8$ con entrada en los números enteros del 0 al 16. Más información en \cite{optdigits.names}. 

\subsection{Preprocesado}

El preprocesado de los datos es la parte más importante del pipeline en un proyecto de ciencia de datos. De él se espera refinar, ajustar, completar y, en definitiva, mejorar la congruencia y consistencia de los mismos para conseguir mejores resultados en la parte de análisis y clasificación. Para conseguir un preprocesado más acertado, me baso continuamente en distintas visualizaciones que arrojen pistas sobre los pasos a seguir. Propongo los siguientes apartados:

\subsubsection{Balanceo de las clases} 

Un dataset balanceado es primordial para garantizar un correcto aprendizaje del modelo. En caso de desbalanceo, las clases más representadas tendrían un peso mayor a la hora de etiquetar instancias nuevas en test, de forma que las menos representadas acabarían, con gran probabilidad, mal clasificadas. Cuando se da desbalanceo hay dos posibles alternativas: eliminar instancias de las clases más repetidas (undersampling) o generar nuevas de las clases minoritarias (oversampling). En este último caso, se suelen utilizar algoritmos como SMOTE (\cite{smote}). \\

En nuestro caso, las clases están absolutamente balanceadas:

\begin{table}[H]
	\centering
	\begin{tabular}{|c|c|}
		\hline
		Etiqueta & Número de instancias \\ \hline
		0        & 178                  \\ \hline
		1        & 182                  \\ \hline
		2        & 177                  \\ \hline
		3        & 183                  \\ \hline
		4        & 181                  \\ \hline
		5        & 182                  \\ \hline
		6        & 181                  \\ \hline
		7        & 179                  \\ \hline
		8        & 174                  \\ \hline
		9        & 180                  \\ \hline
	\end{tabular}
\end{table}

Veámoslo también gráficamente:

\begin{figure}[H] %con el [H] le obligamos a situar aquí la figura
	\centering
	\includegraphics[scale=0.6]{count-clases.png}  %el parámetro scale permite agrandar o achicar la imagen. En el nombre de archivo puede especificar directorios
	\caption{Histograma con el número de instancias por clase} 
	\label{fig:clases}
\end{figure}

Por tanto, no es necesario hacer ninguna modificación en ese sentido.

\subsubsection{Selección de características: Variabilidad de los datos}

A continuación, estudiamos la calidad de las características (columnas en la matriz). Para ello, realizo una descripción estadística de los datos. De las 63 características se estudia la variabilidad de cada individuo a través de la media, la desviación típica, el recorrido intercuartílico, máximo, mínimo... Represento la matriz de correlación para estudiar la correlación lineal entre las características:

\begin{figure}[H] %con el [H] le obligamos a situar aquí la figura
	\centering
	\includegraphics[scale=0.8]{corr-matrix.png}  %el parámetro scale permite agrandar o achicar la imagen. En el nombre de archivo puede especificar directorios
	\caption{Matriz de correlación de características} 
	\label{fig:corr-mat}
\end{figure}


\section{Airfoil self noise}
\newpage
\section{Bibliografía}

%------------------------------------------------

\bibliography{citas} %archivo citas.bib que contiene las entradas 
\bibliographystyle{plain} % hay varias formas de citar

\end{document}
